\documentclass{book}
\usepackage[a4paper,
            left=0.5in,
            right=0.5in,
            top=0.5in,
            bottom=0.5in,
            ]{geometry}
\usepackage{minted}
\setlength{\parindent}{0pt}
\usepackage{graphicx}
\graphicspath{{.}}

\begin{document}
{\Huge \textbf{Lab 5: Swing and JavaFX}}
\\
\par
{
    \large
    \textbf{Objectives:}
    \begin{itemize}
        \item Use Different layouts in Swing and JavaFX
        \item Learn GUI controls in Swing and JavaFX
        \item Learn about event handling and listener Interfaces
    \end{itemize}

    \textbf{Programs:}
    \begin{enumerate}
        \item \textbf{Program 1 School Management System}
        \begin{itemize}
            \item react/Reactive.java
            \inputminted{java}{react/Reactive.java}
            \item react/Clearable.java
            \inputminted{java}{react/Clearable.java}
            \item react/Renderable.java
            \inputminted{java}{react/Renderable.java}
            \item RegistrationForm.java
            \inputminted{java}{RegistrationForm.java}
        
            \par

            \textbf{Output:}
            \begin{itemize} 
                \item { \textbf{Main Frame}
                    \center{\includegraphics[width=0.5\textwidth]{MainFrame.png}}
                }
                \item{ \textbf{Registration Frame}
                    \center{\includegraphics[width=0.5\textwidth]{RegistrationFrame.png}}
                }
                \item{ \textbf{Registration Error}
                    \center{\includegraphics[width=0.5\textwidth]{RegistrationError.png}}
                }
                \item{ \textbf{View Frame}
                    \center{\includegraphics[width=0.5\textwidth]{ViewFrame.png}}
                }
            \end{itemize}


        \end{itemize}

        \item \textbf{Program 2: JavaFX program}

        \begin{minted}{java}
            package org.mdhe.jfx;
            import javafx.application.Application;
            import javafx.geometry.Insets;
            import javafx.geometry.Pos;
            import javafx.scene.Node;
            import javafx.scene.Parent;
            import javafx.scene.Scene;
            import javafx.scene.control.Button;
            import javafx.scene.control.Tab;
            import javafx.scene.control.TabPane;
            import javafx.scene.control.TextField;
            import javafx.scene.image.Image;
            import javafx.scene.image.ImageView;
            import javafx.scene.layout.*;
            import javafx.scene.paint.Color;
            import javafx.scene.text.Text;
            import javafx.stage.Stage;



            public class HelloApplication extends Application {
                @Override
                public void start(Stage stage){
                    stage.setTitle("WOW");

                    Scene newScene = new Scene(createContent(), 300, 300);

                    stage.setScene(newScene);

                    stage.show();
                }

                private Parent createContent() {
                    Tab BMItab = new Tab("BMI", createBMITab());

                    BMItab.setClosable(false);
                    Tab InterestTab = new Tab("Interest", createInterestTab());
                    InterestTab.setClosable(false);
                    Tab ImageTab = new Tab("Image",createImageTab());
                    ImageTab.setClosable(false);

                    TabPane tabPane = new TabPane(BMItab, InterestTab, ImageTab);

                    return tabPane;

                }

                private Node createBMITab() {


                    TextField height = new TextField();
                    height.setPromptText("Height in meters");
                    TextField weight = new TextField();
                    weight.setPromptText("Weight in KG");
                    Button calculate = new Button("Calculate");

                    Text Result = new Text();
                    Result.setFill(Color.valueOf("green"));
                    calculate.setOnAction(e -> {
                        double h = Double.parseDouble(height.getText());
                        double w = Double.parseDouble(weight.getText());
                        double bmi = w / (h * h);
                        Result.setText(String.valueOf(bmi));
                    });

                    VBox v = new VBox(
                            height,
                            weight,
                            calculate,
                            Result
                    );
                    v.setAlignment(Pos.TOP_CENTER);
                    v.setSpacing(5);


                    VBox details = new VBox(
                            new Text("BMI VALUES"),
                            new Text("Underweight less than 18.5"),
                            new Text("Normal between 18.5 and 24.9"),
                            new Text("Overweight between 25 and 29.9"),
                            new Text("Obese 30 or greater")
                    );
                    details.setSpacing(5);

                    HBox h = new HBox(v,
                            details
                    );
                    h.setSpacing(20);
                    h.setPadding(new Insets(10, 20, 0, 20));

                    return h;
                }

                private Node createInterestTab() {


                    GridPane gridPane = new GridPane(10, 5);

                    gridPane.setPadding(new Insets(10, 20, 0, 20));

                    TextField principal = new TextField();
                    principal.setPromptText("Enter principal");

                    TextField time = new TextField();
                    time.setPromptText("Enter Time");


                    TextField rate = new TextField();
                    rate.setPromptText("Rate");

                    TextField result = new TextField();
                    result.setEditable(false);
                    result.setFocusTraversable(false);

                    Button calculate = new Button("Calculate");
                    calculate.setOnAction(e -> {
                        double p = Double.parseDouble(principal.getText());
                        double t = Double.parseDouble(time.getText());
                        double r = Double.parseDouble(rate.getText());
                        result.setText(String.valueOf((p * t * r) / 100));
                    });


                    gridPane.add(principal, 0, 0);
                    gridPane.add(rate, 0, 1);
                    gridPane.add(time,0,2);
                    gridPane.add(calculate, 0, 3);
                    gridPane.add(result, 1, 0, 1, 4);


                    return gridPane;
                }

                private Node createImageTab(){
                    GridPane grid = new GridPane(10,10);
                    grid.setPadding(new Insets(10,20,0,20));


                    Image img = new Image("file:hojlund.jpg",0,0,true,true);
                    ImageView imgView = new ImageView(img);
                    imgView.setPreserveRatio(true);
                    imgView.setFitWidth(200);
                    grid.add(imgView,0,0);

                    Text hojlundInfo = new Text("Picture of football player Rasmus Hojlund");
                    grid.add(hojlundInfo,1,0);

                    Image java = new Image("https://logos-world.net/wp-content/uploads/2022/07/Java-Logo.png",0,0,true,true);
                    ImageView javaImgView = new ImageView(java);
                    javaImgView.setPreserveRatio(true);
                    javaImgView.setFitWidth(200);
                    grid.add(javaImgView,1,1);

                    Text javaInfo = new Text("Java Logo");
                    grid.add(javaInfo,0,1);


                    return grid;
                }

                public static void main(String[] args) {
                    launch();
                }
            }
        
    \end{minted}
    \par
    \textbf{Output:}
    \begin{itemize} 
        \item { \textbf{BMI Tab}
            \center{\includegraphics[width=0.5\textwidth]{bmiTab.png}}
        }
        \item{ \textbf{Interest Tab}
            \center{\includegraphics[width=0.5\textwidth]{InterestTab.png}}
        }
        \item{ \textbf{Image Tab}
            \center{\includegraphics[width=0.5\textwidth]{imageTab.png}}
        }
    \end{itemize}
    \end{enumerate}
    \par
    \textbf{Conclusion:}
    \begin{itemize}
        \item We learned about different layouts in Swing and JavaFX
        \item We learned about GUI controls in Swing and JavaFX
        \item We learned about event handling and listener Interfaces
    \end{itemize}
}

\end{document}
