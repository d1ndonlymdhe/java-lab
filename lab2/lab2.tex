\documentclass{book}
\usepackage[a4paper,
            left=0.5in,
            right=0.5in,
            top=0.5in,
            bottom=0.5in,
            ]{geometry}
\usepackage{minted}
\usepackage{svg}
\setlength{\parindent}{0pt}
\begin{document}
{\Huge \textbf{Title:} Java Basics}
\\
\\
\\
\par
{\large \textbf{Objectives:}}
\begin{itemize}
    \item{Understand the different types of constructors.}
    \item{Understand reference type and value type.}
    \item{Use Access Specifiers to limit the scope of member attributes and methods.}
    \item{Get familiar with Getter and Setters.}
    \item{Exception Handling}
\end{itemize}
\par
{\large \textbf{Theory:}}
\begin{itemize}
    \item{\textbf{Types of constructors:}}
        constructor is a special method which runs whenever a object is instantiated from a class. They are decalred without a return type and are always public.
        One class can have more than one constructor. There are three types of constructors:
            \begin{enumerate}
                \item{\textbf{Default constructor:}} This type of constructor is defined without any parameters.
                \begin{minted}{java}
                    class MyClass{
                        int a;
                        MyClass(){
                            this.a = 5;
                        }
                    }
                \end{minted}
                \item{\textbf{Parameterized constructor:}} This type of constructor are defined with one or many parameters.
                \begin{minted}{java}
                    class MyClass{
                        int a;
                        MyClass(int a){
                            this.a = a;
                        }
                    }
                \end{minted}
                \item{\textbf{Copy constructor:}} This type of constructor are defined with a parameter which accepts the object of the same class.
                \begin{minted}{java}
                    class MyClass{
                        int a;
                        MyClass(MyClass M){
                            this.a = m.a;
                        }
                    }
                \end{minted}
            \end{enumerate}
    \item{\textbf{Reference types:}}
        When arguments are passed to a method they are either passed as value or reference. In C/C++ the programmer manually sets wether to pass a reference or value.
        But in java it is determined automatically by the type of argument. When argument is passed as refernce the pointer to the object is passed and changes to inner scope is also applied to outer scope.
        When argument is passed as value a copy is made so the changes of inner scope is not seen in outer scope. In java the primitive types are int,float,char and boolean and reference types are ClassTypes,ArrayTypes or TypeVariable. 
    \item{\textbf{Setters and Getters:}}
        In OOP we can set visibility of fields and methods to be only accessible from the class or accessible from anywhere. In java there are two visibility public, visible anywhere, private, visible only inside the class.
        TO access the private members we use getters and setters. They must be public and by convention are declared as getVariable and setVariable.
        \begin{minted}{java}
            class MyClass{
                private int a;
                public int getA(){
                    return this.a;
                }
                public void setA(int a){
                    this.a = a;
                }
            }
        \end{minted}
    \item{\textbf{Exception handling:}}
        Exception are runtime errors. They need to be handled properly for code safety. \verb|try..catch..finally| expression is used for exception handling in Java.
        The code that can throw an Exception is wrapped by the try block, when a run time error occurs it thrown an instance of Exception class or a class which extends the Exception class.
        The catch block is executed if an exception occurs. The finally block is optional it runs in any condition.
        \begin{minted}{java}
            try {
                c = a / b;
            } catch (ArithmeticException e) {
            System.out.println("Division error " + e);
            } finally {
                System.out.println("c = " + c);
            }
        \end{minted}
\end{itemize}
\newpage
\par
\textbf{Program 1: Program to demonstrate that default constructor is created by compiler if no constructors are explicitly defined.}
\inputminted{java}{DefaultConstructor.java}
\par
\textbf{Output:}
0
\\
\\
\par
\textbf{Program 2: Program to test default constructor is not  created automatically if any other constructor is defined explicitly.}
\inputminted{java}{NoDefault.java}
\par
\textbf{Output:}
\begin{minted}{bash}
    mdhe@mdheKoLaptop:~/coding/java/lab2$ javac NoDefault.java
    NoDefault.java:8: error: constructor NoDefault in class NoDefault cannot be applied to given types;
            NoDefault nd = new NoDefault();
                       ^
        required: int
        found:    no arguments
        reason: actual and formal argument lists differ in length
    1 error
\end{minted}
\par
\textbf{Program 3: Program to demonstrate the use of all types of constructors.}
\inputminted{java}{AllConstructors.java}
\par
\textbf{Output:}
\begin{verbatim}
Roll = 0 Name =
Roll = 4 Name = hari
Roll = 4 Name = hari 
\end{verbatim}
\par
\textbf{Program 4: Program to demonstrate reference type with copy constructor:}
\inputminted{java}{CopyConstructor.java}
\par
\textbf{Output:}
\begin{verbatim}
50.56
Room 1 
Area = 50.56 copied = true
Room 2
Area = 50.56 copied = false
\end{verbatim}
\par
\textbf{Program 5: Program to demonstrate the use of access specifier and how getter and setter helps to access the private attributes.}
\inputminted{java}{AccessSpecifier.java}
\par
\textbf{Output:}
\begin{verbatim}
Enter username
MyBadWord
Do not include bad words
Enter username
ayush
Enter password
abcd
Minimum password length is 8
Enter password
abcdefghij
id = 0 username = ayush password = abcdefghij
\end{verbatim}
\par
\textbf{Program 6: Program to demonstrate the use of exception handling}
\inputminted{java}{ExceptionHandling.java}
\par
\textbf{Output:}
\begin{verbatim}
Enter value for a
5
Enter value for b
e
Unhandled input java.lang.NumberFormatException: For input string: "e"
Set b as 0
/ by zero
c = 0.0
\end{verbatim}
\par
\textbf{Program 7: Program to creeate a Time class with various attributes and methods}
\\
\includesvg{classDiagram.svg}
\inputminted{java}{Lab2.java}
\textbf{Output:}
\begin{verbatim}
15:05:07
15:05:07
15:06:00
Second should be > 0
Finally called
05:52:11
01:40:29
00:08:20
6
\end{verbatim}
\par
\textbf{Conclusion:} We learned about different types of constructors, refernce types, copy constructors, Access Specifiers and Error handling.
\par
\end{document}
